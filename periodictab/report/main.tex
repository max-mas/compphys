\documentclass[a4paper,DIV=12,english]{scrartcl}
\usepackage[utf8]{inputenc}
\usepackage{fancyhdr}
\usepackage{bookmark}
\usepackage{graphicx}
\usepackage{hyperref}
\usepackage{xurl}
\usepackage[sorting=none, style=numeric-comp]{biblatex}
\addbibresource{ref.bib}
\usepackage{csquotes}
\usepackage[dvipsnames]{xcolor}
\usepackage[num]{isodate}
\usepackage{amsthm}
\usepackage{amssymb}
\usepackage{bbm}
\usepackage{amsmath}
\usepackage{tikz}
%\usepackage{pgfplots}
    %\usepgfplotslibrary{fillbetween}
\usepackage{svg}
\usepackage{braket}
\usepackage{caption}
\usepackage{subcaption}
\usepackage{placeins}
%\setlength\parindent{0pt}
\usepackage{wrapfig}
\usepackage{float}


% Fakesection
\newcommand{\fakesection}[1]{%
    \par\refstepcounter{section}                                        % Increase section counter
    \sectionmark{#1}                                                    % Add section mark (header)
    \addcontentsline{toc}{section}{\protect\numberline{\thesection}#1}  % Add section to ToC
    % Add more content here, if needed.
} 

\renewcommand{\footrulewidth}{0.5pt}
\pagestyle{fancy}
\fancyhf{}
\fancyhead[L]{\leftmark}
\fancyhead[R]{}

\fancyfoot[C]{Computational Physics: Explaining the Periodic Table}
\fancyfoot[R]{\thepage}

\title{Computational Physics: Explaining the Periodic Table}
\author{Stockholm University, Spring Term 2024 \\Max Maschke}
\date{May 14 2024}


\begin{document}
\maketitle


\tableofcontents
\newpage


\newpage
\section{Introduction}\cite{github}

\section{Physics of Atoms and Mean Field Theory}
From a theoretical point of view, an atom is a quantum mechanical system in which electrons are bound by a central nuclear charge which is often approximated as a point charge of magnitude $Ze$. Despite their abundance and the limited number of particles involved, the fermionic nature of electrons is a major hindrance to the treatment of the electronic structure of atoms. As the wave function of a system with $N_e$ electrons is a function of $3N_e$ variables (neglecting the spin degree of freedom), the memory required to represent it on a grid with $m$ points per dimension scales rather unpleasantly as $m^{3N_e}$. For a conservative number of $m=100$ and $N_e=6$ electrons for, e.g., Carbon, $100^{18}=10^{36}$ numbers would have to be stored, which by far overwhelms the collective storage capacity of all computers on Earth. % TODO citation!

Luckily, the full wave function is not actually needed. According to a theorem due to Hohenberg and Kohn~\cite{kohn_hohenberg}, the ground state wave function is a unique functional of the ground state electron density, which is only a function of 3 variables, and vice versa. That is, in principle, all information that can be extracted from the wave function $\psi(\textbf{r}_1,\sigma_1, \dots, \textbf{r}_{N_e},\sigma_{N_e})$ can also be extracted from the density
\begin{equation}
    n(\textbf{r}) = N_e \sum_{\sigma_i}\int_{\mathbb{R}^3} \text{d}^{3(N_e-1)}x\,\left|\psi(\textbf{r},\sigma_1,\textbf{x}_2,\sigma_2,\dots,\textbf{x}_{N_e},\sigma_{N_e})\right|^2.
\end{equation}
This is the motivation behind density functional theory (DFT), which is today arguably the most important computational method in quantum chemistry. In this report, we explore a simple version of this approach that relies on finding a self-consistent single particle potential $v^{ee}(\textbf{r})$ such that the Slater-determinant associated with the single-particle Hamiltonian (in atomic units)
\begin{equation}
    \mathcal{H} = -\frac{1}{2}\partial_\textbf{x}^2 - \frac{Z}{r} + V^{ee}(r)
\end{equation}
approximates the ground state of the full $N_e$-particle Hamiltonian. The ansatz for $V^{ee}$ includes two terms. The first is the direct electrostatic potential 



\section{Implementation and Numerics}

\section{Results}

\newpage
\fakesection{References}
\printbibliography


\end{document}
